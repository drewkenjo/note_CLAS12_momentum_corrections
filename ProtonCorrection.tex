\section{Proton momentum correction
\\\small\color{red} Andrey, Timothy, Nick
}


\begin{itemize}
    \item Andrey: extract proton correction from $ep\pi^0$
    \item Nick: apply and validate proton correction to $eP\pi^+(\pi^-)$ and $eP\pi^-(\pi^+)$
    \item Timothy: apply and validate proton correction using his sample
\end{itemize}




The $ep\rightarrow ep\pi^0$ channel may be used to monitor the effect of electron momentum correction on independent event sample, as shown on Fig.~\ref{eppi0.ecorr.mm2vspe.graph1}.
For each electron momentum bin the 1D distribution of $MM^2_{epX}$ values were fitted with a gaussian signal and linear background to extract the peak position of $MM^2$ as a function of electron momentum.
The procedure was performed individually for each electron sector and the values are compared to the nominal expected value of square pion mass.
The improvement is the most noticeable for outbending dataset where electron correction alone adjusts $MM^2$ peak positions closer to the expected value.
For inbending dataset the $MM^2$ positions seem to be affected very slightly except in sector 3 and 4.
Note, however, that $MM^2_{epX}$ quantity depends on proton momentum as well, and therefore we should not attempt to constrain is using only electron momentum correction.
Therefore, using the fixed electron corrections $MM^2_{epX}$ distributions can be used to develop the data-driven proton momentum corrections, as described in Section~\ref{pcorr.section}.





\begin{figure}
	\begin{tcolorbox}[halign=center,colback=white,colbacktitle=black!40!white,colframe=black!80!white,left=0pt,right=0pt,top=0pt,bottom=0pt,boxrule=0.3pt,title=\scriptsize\bfseries Inbending]
		\includegraphics[width=\linewidth]{pdf/eppi0.ecorr.mm2vspe.graph1.inb.pdf}
	\end{tcolorbox}
	\vspace{-6pt}
	\begin{tcolorbox}[halign=center,colback=white,colbacktitle=black!40!white,colframe=black!80!white,left=0pt,right=0pt,top=0pt,bottom=0pt,boxrule=0.3pt,title=\scriptsize\bfseries Outbending]
		\includegraphics[width=\linewidth]{pdf/eppi0.ecorr.mm2vspe.graph1.outb.pdf}
	\end{tcolorbox}
	\vspace{-6pt}
	\caption{For each electron momentum bin the peak position of $MM^2_{epX}$ distribution is plotted vs the mean value of electron momenta before (black) and after (red) electron momentum corrections applied. The improvement is most noticeable for outbending data, while for inbending data the $MM^2$ positions seem to be affected very slightly except in sector 3 and 4. Note that $MM^2_{epX}$ quantity depends on proton momentum as well, and therefore we can not expect to constrain is using only electron momentum correction.}
	\label{eppi0.ecorr.mm2vspe.graph1}
\end{figure}
\FloatBarrier


