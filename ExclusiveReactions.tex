\section{Exclusive reactions}

\subsection{$e\pi^+(N)$ exclusive channel
\\\small\color{red} Utsav, Richard
}

\subsubsection{Kinematics}

It is essential to look at the kinematic coverage of the particles in the reaction channel. Fig.~\ref{fig:in_out_ePipX_pthlfie} shows the distribution of electron momentum ($p_e$), electron polar angle ($\theta_e$), and electron azimuthal angle ($\phi_e$) in two dimensional plots. In order to maintain local $\phi_e$ distribution symmetry around 0, a shifted $\phi_e$ value is used throughout the analysis, such that,
\begin{align*}
    \text{shifted}\,\phi_e = \text{unshifted}\,\phi_e - 30/p_e,
\end{align*}
as shown in Fig.~\ref{fig:in_out_ePipX_pthlfie_shifted}.

\begin{figure}[h]
	\begin{tcolorbox}[halign=center,colback=white,colbacktitle=black!40!white,colframe=black!80!white,left=0pt,right=0pt,top=0pt,bottom=0pt,boxrule=0.3pt,title=\scriptsize\bfseries Inbending]
		\includegraphics[width=0.9\linewidth]{pictures/utsav/in_ePipX_pthlfie.png}
	\end{tcolorbox}
	\vspace{-8pt}
	\begin{tcolorbox}[halign=center,colback=white,colbacktitle=black!40!white,colframe=black!80!white,left=0pt,right=0pt,top=0pt,bottom=0pt,boxrule=0.3pt,title=\scriptsize\bfseries Outbending]
		\includegraphics[width=0.9\linewidth]{pictures/utsav/out_ePipX_pthlfie.png}
	\end{tcolorbox}
	\vspace{-8pt}
	\caption{Two dimensional plot showing distribution of electron momentum ($p_e$), electron polar angle ($\theta_e$), and electron azimuthal angle ($\phi_e$) in $e\pi^+X$ channel for the inbending and outbending data.}
	\label{fig:in_out_ePipX_pthlfie}
\end{figure}

\begin{figure}[h]
	\begin{tcolorbox}[halign=center,colback=white,colbacktitle=black!40!white,colframe=black!80!white,left=0pt,right=0pt,top=0pt,bottom=0pt,boxrule=0.3pt,title=\scriptsize\bfseries Inbending]
		\includegraphics[width=0.9\linewidth]{pictures/utsav/in_ePipX_pthlfie_shifted.png}
	\end{tcolorbox}
	\vspace{-8pt}
	\begin{tcolorbox}[halign=center,colback=white,colbacktitle=black!40!white,colframe=black!80!white,left=0pt,right=0pt,top=0pt,bottom=0pt,boxrule=0.3pt,title=\scriptsize\bfseries Outbending]
		\includegraphics[width=0.9\linewidth]{pictures/utsav/out_ePipX_pthlfie_shifted.png}
	\end{tcolorbox}
	\vspace{-8pt}
	\caption{Two dimensional plot showing distribution of electron momentum ($p_e$), electron polar angle ($\theta_e$), and shifted electron azimuthal angle ($\phi_e$) for in $e\pi^+X$ channel the inbending and outbending data.}
	\label{fig:in_out_ePipX_pthlfie_shifted}
\end{figure}

\FloatBarrier

Fig.~\ref{fig:in_out_ePipX_pthlfipip} shows the distribution of pion momentum ($p_{\pi^+}$), pion polar angle ($\theta_{\pi^+}$), and pion azimuthal angle ($\phi_{\pi^+}$) in two dimensional plots. In order to maintain local $\phi_{\pi^+}$ distribution symmetry around 0, a shifted $\phi_{\pi^+}$ value is used throughout the analysis, such that,
\begin{align*}
    \text{shifted}\,\phi_{\pi^+} = \text{unshifted}\,\phi_{\pi^+} + 32/(p_{\pi^+} + 0.05),
\end{align*}
as shown in Fig.~\ref{fig:in_out_ePipX_pthlfipip_shifted}.

\begin{figure}[h]
	\begin{tcolorbox}[halign=center,colback=white,colbacktitle=black!40!white,colframe=black!80!white,left=0pt,right=0pt,top=0pt,bottom=0pt,boxrule=0.3pt,title=\scriptsize\bfseries Inbending]
		\includegraphics[width=0.9\linewidth]{pictures/utsav/in_ePipX_pthlfipip.png}
	\end{tcolorbox}
	\vspace{-8pt}
	\begin{tcolorbox}[halign=center,colback=white,colbacktitle=black!40!white,colframe=black!80!white,left=0pt,right=0pt,top=0pt,bottom=0pt,boxrule=0.3pt,title=\scriptsize\bfseries Outbending]
		\includegraphics[width=0.9\linewidth]{pictures/utsav/out_ePipX_pthlfipip.png}
	\end{tcolorbox}
	\vspace{-8pt}
	\caption{Two dimensional plot showing distribution of $\pi^+$ momentum ($p_{\pi^+}$), $\pi^+$ polar angle ($\theta_{\pi^+}$), and $\pi^+$ azimuthal angle ($\phi_{\pi^+}$) in $e\pi^+X$ channel for the inbending and outbending data.}
	\label{fig:in_out_ePipX_pthlfipip}
\end{figure}

\begin{figure}[h]
	\begin{tcolorbox}[halign=center,colback=white,colbacktitle=black!40!white,colframe=black!80!white,left=0pt,right=0pt,top=0pt,bottom=0pt,boxrule=0.3pt,title=\scriptsize\bfseries Inbending]
		\includegraphics[width=0.9\linewidth]{pictures/utsav/in_ePipX_pthlfipip_shifted.png}
	\end{tcolorbox}
	\vspace{-8pt}
	\begin{tcolorbox}[halign=center,colback=white,colbacktitle=black!40!white,colframe=black!80!white,left=0pt,right=0pt,top=0pt,bottom=0pt,boxrule=0.3pt,title=\scriptsize\bfseries Outbending]
		\includegraphics[width=0.9\linewidth]{pictures/utsav/out_ePipX_pthlfipip_shifted.png}
	\end{tcolorbox}
	\vspace{-8pt}
	\caption{Two dimensional plot showing distribution of $\pi^+$ momentum ($p_{\pi^+}$), $\pi^+$ polar angle ($\theta_{\pi^+}$), and shifted $\pi^+$ azimuthal angle ($\phi_{\pi^+}$) in $e\pi^+X$ channel for the inbending and outbending data.}
	\label{fig:in_out_ePipX_pthlfipip_shifted}
\end{figure}


\FloatBarrier
\subsubsection{Missing mass $e\pi^+X$}

The missing mass ($MM$) distribution for the $e\pi^+X$ channel shows neutron peak. The distribution for all six sectors for the inbending data is shown in Fig.~\ref{fig:in_out_ePipX_MM_sectors}. The 2D plot of the $MM(e\pi^+X)$ and electron momentum ($p_e$) for all sectors is shown in Fig.~\ref{fig:in_out_ePipX_MM0vspe_sectors}. These figures do not give much of an idea about the need for the momentum corrections. One needs to look deeper into the momentum dependence of the $MM$ plots.

\begin{figure}[h]
	\begin{tcolorbox}[halign=center,colback=white,colbacktitle=black!40!white,colframe=black!80!white,left=0pt,right=0pt,top=0pt,bottom=0pt,boxrule=0.3pt,title=\scriptsize\bfseries Inbending]
		\includegraphics[width=0.9\linewidth]{pictures/utsav/in_ePipX_MM_sectors.png}
	\end{tcolorbox}
	\vspace{-8pt}
	\begin{tcolorbox}[halign=center,colback=white,colbacktitle=black!40!white,colframe=black!80!white,left=0pt,right=0pt,top=0pt,bottom=0pt,boxrule=0.3pt,title=\scriptsize\bfseries Outbending]
		\includegraphics[width=0.9\linewidth]{pictures/utsav/out_ePipX_MM_sectors.png}
	\end{tcolorbox}
	\vspace{-8pt}
	\caption{1D integrated Missing mass, $MM(e\pi^+X)$ distribution shown for each sector for inbending and outbending data.}
	\label{fig:in_out_ePipX_MM_sectors}
\end{figure}

\begin{figure}[h]
	\begin{tcolorbox}[halign=center,colback=white,colbacktitle=black!40!white,colframe=black!80!white,left=0pt,right=0pt,top=0pt,bottom=0pt,boxrule=0.3pt,title=\scriptsize\bfseries Inbending]
		\includegraphics[width=0.9\linewidth]{pictures/utsav/in_ePipX_MM0vspe_sectors.png}
	\end{tcolorbox}
	\vspace{-8pt}
	\begin{tcolorbox}[halign=center,colback=white,colbacktitle=black!40!white,colframe=black!80!white,left=0pt,right=0pt,top=0pt,bottom=0pt,boxrule=0.3pt,title=\scriptsize\bfseries Outbending]
		\includegraphics[width=0.9\linewidth]{pictures/utsav/out_ePipX_MM0vspe_sectors.png}
	\end{tcolorbox}
	\vspace{-8pt}
	\caption{2D distribution for the Missing mass, $MM(e\pi^+X)$ and electron momentum ($p_e$) are shown for each sector for inbending and outbending data.}
	\label{fig:in_out_ePipX_MM0vspe_sectors}
\end{figure}

\begin{figure}[h]
	\begin{tcolorbox}[halign=center,colback=white,colbacktitle=black!40!white,colframe=black!80!white,left=0pt,right=0pt,top=0pt,bottom=0pt,boxrule=0.3pt,title=\scriptsize\bfseries Inbending]
		\includegraphics[width=0.9\linewidth]{pictures/utsav/in_ePipX_MM0vsppip_sectors.png}
	\end{tcolorbox}
	\vspace{-8pt}
	\begin{tcolorbox}[halign=center,colback=white,colbacktitle=black!40!white,colframe=black!80!white,left=0pt,right=0pt,top=0pt,bottom=0pt,boxrule=0.3pt,title=\scriptsize\bfseries Outbending]
		\includegraphics[width=0.9\linewidth]{pictures/utsav/out_ePipX_MM0vsppip_sectors.png}
	\end{tcolorbox}
	\vspace{-8pt}
	\caption{2D distribution for the Missing mass, $MM(e\pi^+X)$ and $\pi^+$ momentum ($p_{\pi^+}$) are shown for each sector for inbending and outbending data.}
	\label{fig:in_out_ePipX_MM0vsppip_sectors}
\end{figure}


The $MM$ distribution in each sector is now studied in different $p_e$ bins. Slices of momentum bins are taken so that each 1D $MM$ distribution are now fitted using a Gaussian signal function and a linear background function. After subtracting background, the peak position is plotted as a functions of those electron momentum bins. The $MM$ positions are compared with the expected missing mass peak, the neutron peak in this channel, and is shown in Fig.~\ref{fig:in_out_ePipX_MM0vspe}. The figure clearly shows the discrepancy of the $MM$ position away from the expected neutron peak position. These plots demonstrate the importance of momentum corrections.


\begin{figure}[h]
	\begin{tcolorbox}[halign=center,colback=white,colbacktitle=black!40!white,colframe=black!80!white,left=0pt,right=0pt,top=0pt,bottom=0pt,boxrule=0.3pt,title=\scriptsize\bfseries Inbending]
		\includegraphics[width=0.9\linewidth]{pictures/utsav/in_ePipX_MM0vspe.png}
	\end{tcolorbox}
	\vspace{-8pt}
	\begin{tcolorbox}[halign=center,colback=white,colbacktitle=black!40!white,colframe=black!80!white,left=0pt,right=0pt,top=0pt,bottom=0pt,boxrule=0.3pt,title=\scriptsize\bfseries Outbending]
		\includegraphics[width=0.9\linewidth]{pictures/utsav/out_ePipX_MM0vspe.png}
	\end{tcolorbox}
	\vspace{-8pt}
	\caption{Missing mass, $MM(e\pi^+X)$, as a function of electron momentum $p_e$ for the inbending and outbending data. The red line represents the expected $MM$ peak, which is the neutron peak position.}
	\label{fig:in_out_ePipX_MM0vspe}
\end{figure}

\begin{figure}[h]
	\begin{tcolorbox}[halign=center,colback=white,colbacktitle=black!40!white,colframe=black!80!white,left=0pt,right=0pt,top=0pt,bottom=0pt,boxrule=0.3pt,title=\scriptsize\bfseries Inbending]
		\includegraphics[width=0.9\linewidth]{pictures/utsav/in_ePipX_MM0vsppip.png}
	\end{tcolorbox}
	\vspace{-8pt}
	\begin{tcolorbox}[halign=center,colback=white,colbacktitle=black!40!white,colframe=black!80!white,left=0pt,right=0pt,top=0pt,bottom=0pt,boxrule=0.3pt,title=\scriptsize\bfseries Outbending]
		\includegraphics[width=0.9\linewidth]{pictures/utsav/out_ePipX_MM0vsppip.png}
	\end{tcolorbox}
	\vspace{-8pt}
	\caption{Missing mass, $MM(e\pi^+X)$, as a function of $\pi^+$ momentum $p_{\pi^+}$ for the inbending and outbending data. The red line represents the expected $MM$ peak, which is the neutron peak position.}
	\label{fig:in_out_ePipX_MM0vsppip}
\end{figure}


\FloatBarrier

\subsection{$ep\pi^+\pi^-$ exclusive channel
\\\small\color{red} Utsav, Nick
}

\subsubsection{Kinematics}

\begin{figure}[!ht]
\centering
\includegraphics[width=\textwidth]{pictures/utsav/in_ePipPimX_pthlfie.png} 
\caption{Two dimensional plot showing distribution of electron momentum ($p_e$), electron polar angle ($\theta_e$), and electron azimuthal angle ($\phi_e$) in $e\pi^+\pi^-X$ channel for the inbending data.} \label{fig:in_ePipPimX_pthlfie}
\end{figure}

\begin{figure}[!ht]
\centering
\includegraphics[width=\textwidth]{pictures/utsav/in_ePipPimX_pthlfie_shifted.png} 
\caption{Two dimensional plot showing distribution of electron momentum ($p_e$), electron polar angle ($\theta_e$), and shifted electron azimuthal angle ($\phi_e$) for in $e\pi^+\pi^-X$ channel the inbending data.} \label{fig:in_ePipPimX_pthlfie_shifted}
\end{figure}

\FloatBarrier

\subsubsection{Missing mass $e\pi^+\pi^-X$}

\begin{figure}[!ht]
\centering
\includegraphics[width=\textwidth]{pictures/utsav/in_out_ePipPimX_MMvspe.png} 
\caption{Missing mass, $MM(e\pi^+\pi^-X)$, as a function of electron momentum $p_e$ for both inbending and outbending data.} \label{fig:in_out_ePipX_MMvspe}
\end{figure}

\begin{figure}[!ht]
\centering
\includegraphics[width=\textwidth]{pictures/utsav/in_out_ePipPimX_MMvsppip.png} 
\caption{Missing mass, $MM(e\pi^+\pi^-X)$, as a function of pion momentum $p_{\pi^+}$ for both inbending and outbending data.} \label{fig:in_out_ePipX_MMvspe}
\end{figure}
\FloatBarrier

\subsection{$ep\pi^0$ exclusive channel
\\\small\color{red} Andrey
}

\subsubsection{Kinematics}

\subsubsection{Missing mass squared $epX$}



On Fig.~\ref{eppi0.ecorr.mm2vspe} the 2D distributions of $MM^2$ for $ep\rightarrow epX$ system vs electron momentum are plotted for exclusive deeply virtual $\pi^0$ events from experimental data for each electron sector individually.
It is difficult to make conclusive statements based on the 2D plots due to resolution effects widening the $MM^2$ distributions.
To study the dependence of $MM^2$ values on electron momentum the data are split into multiple electron momentum bins, and $MM^2$ 1D distributions in each of the bin are analyzed and fitted using the function consisting of of gaussian signal and linear background.
The peak position values are extracted from the fit parameters and plotted vs momentum bin values as shown on Fig.~\ref{eppi0.ecorr.mm2vspe.graph0}.
These graphs clearly show the deviations of $MM^2$ peak positions from their expected values and support the dependence of correction parameters on electron kinematics.

\begin{figure}[h]
	\begin{tcolorbox}[halign=center,colback=white,colbacktitle=black!40!white,colframe=black!80!white,left=0pt,right=0pt,top=0pt,bottom=0pt,boxrule=0.3pt,title=\scriptsize\bfseries Inbending]
		\includegraphics[width=0.9\linewidth]{pdf/eppi0.ecorr.mm2vspe.inb.pdf}
	\end{tcolorbox}
	\vspace{-8pt}
	\begin{tcolorbox}[halign=center,colback=white,colbacktitle=black!40!white,colframe=black!80!white,left=0pt,right=0pt,top=0pt,bottom=0pt,boxrule=0.3pt,title=\scriptsize\bfseries Outbending]
		\includegraphics[width=0.9\linewidth]{pdf/eppi0.ecorr.mm2vspe.outb.pdf}
	\end{tcolorbox}
	\vspace{-8pt}
	\caption{The 2D distributions of $MM^2$ of $ep\rightarrow epX$ system vs electron momentum are plotted for each electron sector using exclusive deeply virtual $\pi^0$ electroproduction events.}
	\label{eppi0.ecorr.mm2vspe}
\end{figure}
\begin{figure}[h]
	\begin{tcolorbox}[halign=center,colback=white,colbacktitle=black!40!white,colframe=black!80!white,left=0pt,right=0pt,top=0pt,bottom=0pt,boxrule=0.3pt,title=\scriptsize\bfseries Inbending]
		\includegraphics[width=0.9\linewidth]{pdf/eppi0.ecorr.mm2vspe.graph0.inb.pdf}
	\end{tcolorbox}
	\vspace{-8pt}
	\begin{tcolorbox}[halign=center,colback=white,colbacktitle=black!40!white,colframe=black!80!white,left=0pt,right=0pt,top=0pt,bottom=0pt,boxrule=0.3pt,title=\scriptsize\bfseries Outbending]
		\includegraphics[width=0.9\linewidth]{pdf/eppi0.ecorr.mm2vspe.graph0.outb.pdf}
	\end{tcolorbox}
	\vspace{-8pt}
	\caption{For each electron momentum bin the peak position of $MM^2_{epX}$ distribution is plotted vs the mean value of electron momenta in each electron sector for the exclusive deeply virtual $\pi^0$ electroproduction events.}
	\label{eppi0.ecorr.mm2vspe.graph0}
\end{figure}


\FloatBarrier