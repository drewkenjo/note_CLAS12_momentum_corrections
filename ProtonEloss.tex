\chapter{Proton energy loss
\\\small\color{red} Andrey, Timothy
}


\begin{itemize}
    \item validate our studies with Pierre and Sangbaek
    \item protons from SIDIS sample (Timothy)
\end{itemize}


\begin{figure}[h]
	\includegraphics[width=0.95\linewidth]{pdf/proton.eloss.dpvsp.fits.pdf}
	\caption{The 2D distributions of $\Delta P=P_{gen}-P_{rec}$ vs proton momentum from CLAS12 GEMC simulation.
	The plots are shown for two polar angle regions that exhibit significantly different energy losses due to the different amount of material along the proton propagation path.
	Red markers are gaussian mean values for each momentum slice, magenta curves are the exponential fits that describe the energy loss value as a function of proton momentum, the red dashed lines are corrections from Pierre's study based on polynomial extrapolation, black scatter plots are correction bands from detailed Sangbaek study.}
	\label{eppi0.ecorr.mm2vspe.graph1}
\end{figure}

Additional studies were performed by Timothy Hayward on the SIDIS sample $ep \rightarrow e' p \pi^+ X$ analyzed from RGA-Fall2018 data using the PID from the RGA Common Analysis note. The only corrections applied here were the ``proton energy loss'' corrections discussed above and derived by Andrey Kim. In the $e p \rightarrow e' p \pi^+ X$ sample it is expected that the missing mass spectrum should contain two clearly visible peaks around the $\pi^-$ and $\rho^-$ masses. Both inbending and outbending data were divided into a number of proton momentum and polar angle bins and the position of the missing mass squared peak corresponding to the $\pi^-$ was extracted with a Guassian fit on top of a background quadratic. Examples of this extraction can be seen in FIG.~\ref{fig:hayward_example_fit}. Inbending distributions and peak positions are shown in FIG.~\ref{fig:hayward_inbending}. Outbending distributions and peak positions are shown in FIG.~\ref{fig:hayward_outbending}. There is significant improvement in the position of the $\pi^-$ peak in inbending data, especially at lower momenta where energy loss is more significant. The outbending data peak position is not quite as precise as the inbending, but significant improvement was observed in removing the momentum-dependence. Finally, all peak positions as a function of momenta for separate polar angle bins are shown in FIG.~\ref{fig:hayward_results}.

\begin{figure}[h!]
\begin{subfigure}[t]{0.49\textwidth}
\centering
\includegraphics[width=\linewidth]{pictures/timothy/inbending_4a.png}
%        \caption{}\label{fig:fig_a}
\end{subfigure}
\begin{subfigure}[t]{0.51\textwidth}
\centering
\includegraphics[width=\linewidth]{pictures/timothy/inbending_4b.png}
%        \caption{}\label{fig:fig_a}
\end{subfigure}

%
%\begin{minipage}[t]{.4\textwidth}
\caption{Example fits of the $\pi^-$ missing mass squared peak for a particular proton momentum bin before (left) and after (right) corrections for Fall2018 inbending data in the process $ep \rightarrow e' p \pi^+ X$. The $\pi^-$ mass squared, 0.0195~GeV$^2$ is shown as a vertical black line, the back ground fit is shown as a blue line, the peak fit is shown as a red line and the total fit is shown as a green line. Improvement is seen in the peak position after proton energy loss corrections.}
\label{fig:hayward_example_fit}
%\end{minipage}
\end{figure}

\begin{figure}[h!]
\begin{subfigure}[t]{0.32\textwidth}
\centering
\includegraphics[width=\linewidth]{pictures/timothy/inbending_0.png}
%        \caption{}\label{fig:fig_a}
\end{subfigure}
\begin{subfigure}[t]{0.32\textwidth}
\centering
\includegraphics[width=\linewidth]{pictures/timothy/inbending_1.png}
%        \caption{}\label{fig:fig_a}
\end{subfigure}
\begin{subfigure}[t]{0.32\textwidth}
\centering
\includegraphics[width=\linewidth]{pictures/timothy/inbending_2.png}
%        \caption{}\label{fig:fig_a}
\end{subfigure}
\begin{subfigure}[t]{0.32\textwidth}
\centering
\includegraphics[width=\linewidth]{pictures/timothy/inbending_3.png}
%        \caption{}\label{fig:fig_a}
\end{subfigure}
\begin{subfigure}[t]{0.32\textwidth}
\centering
\includegraphics[width=\linewidth]{pictures/timothy/inbending_4.png}
%        \caption{}\label{fig:fig_a}
\end{subfigure}
\begin{subfigure}[t]{0.32\textwidth}
\centering
\includegraphics[width=\linewidth]{pictures/timothy/inbending_5.png}
%        \caption{}\label{fig:fig_a}
\end{subfigure}
\begin{subfigure}[t]{0.32\textwidth}
\centering
\includegraphics[width=\linewidth]{pictures/timothy/inbending_6.png}
%        \caption{}\label{fig:fig_a}
\end{subfigure}
\begin{subfigure}[t]{0.32\textwidth}
\centering
\includegraphics[width=\linewidth]{pictures/timothy/inbending_7.png}
%        \caption{}\label{fig:fig_a}
\end{subfigure}
\begin{subfigure}[t]{0.32\textwidth}
\centering
\includegraphics[width=\linewidth]{pictures/timothy/inbending_8.png}
%        \caption{}\label{fig:fig_a}
\end{subfigure}
\begin{subfigure}[t]{0.32\textwidth}
\centering
\includegraphics[width=\linewidth]{pictures/timothy/inbending_9.png}
%        \caption{}\label{fig:fig_a}
\end{subfigure}
\begin{subfigure}[t]{0.32\textwidth}
\centering
\includegraphics[width=\linewidth]{pictures/timothy/inbending_10.png}
%        \caption{}\label{fig:fig_a}
\end{subfigure}
\begin{subfigure}[t]{0.32\textwidth}
\centering
\includegraphics[width=\linewidth]{pictures/timothy/inbending.png}
%        \caption{}\label{fig:fig_a}
\end{subfigure}

%
%\begin{minipage}[t]{.4\textwidth}
\caption{Distributions of the $\pi^-$ missing mass squared peak for a particular proton momentum bin before (black) and after (red) corrections for Fall2018 inbending data in the process $ep \rightarrow e' p \pi^+ X$. The $\pi^-$ mass squared, 0.0195~GeV$^2$, is shown as a vertical black line, the back ground fit is shown as a blue line, the peak fit is shown as a red line and the total fit is shown as a green line. Improvement is seen in the peak position after proton energy loss corrections.}
\label{fig:hayward_inbending}
%\end{minipage}
\end{figure}

\begin{figure}[h!]
\begin{subfigure}[t]{0.32\textwidth}
\centering
\includegraphics[width=\linewidth]{pictures/timothy/outbending_0.png}
%        \caption{}\label{fig:fig_a}
\end{subfigure}
\begin{subfigure}[t]{0.32\textwidth}
\centering
\includegraphics[width=\linewidth]{pictures/timothy/outbending_1.png}
%        \caption{}\label{fig:fig_a}
\end{subfigure}
\begin{subfigure}[t]{0.32\textwidth}
\centering
\includegraphics[width=\linewidth]{pictures/timothy/outbending_2.png}
%        \caption{}\label{fig:fig_a}
\end{subfigure}
\begin{subfigure}[t]{0.32\textwidth}
\centering
\includegraphics[width=\linewidth]{pictures/timothy/outbending_3.png}
%        \caption{}\label{fig:fig_a}
\end{subfigure}
\begin{subfigure}[t]{0.32\textwidth}
\centering
\includegraphics[width=\linewidth]{pictures/timothy/outbending_4.png}
%        \caption{}\label{fig:fig_a}
\end{subfigure}
\begin{subfigure}[t]{0.32\textwidth}
\centering
\includegraphics[width=\linewidth]{pictures/timothy/outbending_5.png}
%        \caption{}\label{fig:fig_a}
\end{subfigure}
\begin{subfigure}[t]{0.32\textwidth}
\centering
\includegraphics[width=\linewidth]{pictures/timothy/outbending_6.png}
%        \caption{}\label{fig:fig_a}
\end{subfigure}
\begin{subfigure}[t]{0.32\textwidth}
\centering
\includegraphics[width=\linewidth]{pictures/timothy/outbending_7.png}
%        \caption{}\label{fig:fig_a}
\end{subfigure}
\begin{subfigure}[t]{0.32\textwidth}
\centering
\includegraphics[width=\linewidth]{pictures/timothy/outbending_8.png}
%        \caption{}\label{fig:fig_a}
\end{subfigure}
\begin{subfigure}[t]{0.32\textwidth}
\centering
\includegraphics[width=\linewidth]{pictures/timothy/outbending_9.png}
%        \caption{}\label{fig:fig_a}
\end{subfigure}
\begin{subfigure}[t]{0.32\textwidth}
\centering
\includegraphics[width=\linewidth]{pictures/timothy/outbending_10.png}
%        \caption{}\label{fig:fig_a}
\end{subfigure}
\begin{subfigure}[t]{0.32\textwidth}
\centering
\includegraphics[width=\linewidth]{pictures/timothy/outbending.png}
%        \caption{}\label{fig:fig_a}
\end{subfigure}

%
%\begin{minipage}[t]{.4\textwidth}
\caption{Distributions of the $\pi^-$ missing mass squared peak for a particular proton momentum bin before (black) and after (red) corrections for Fall2018 outbending data in the process $ep \rightarrow e' p \pi^+ X$. The $\pi^-$ mass squared, 0.0195~GeV$^2$, is shown as a vertical black line, the back ground fit is shown as a blue line, the peak fit is shown as a red line and the total fit is shown as a green line. Improvement is seen in the peak position after proton energy loss corrections.}
\label{fig:hayward_outbending}
%\end{minipage}
\end{figure}

\begin{figure}[h!]
\begin{subfigure}[t]{0.32\textwidth}
\centering
\includegraphics[width=\linewidth]{pictures/timothy/inbending.png}
%        \caption{}\label{fig:fig_a}
\end{subfigure}
\begin{subfigure}[t]{0.32\textwidth}
\centering
\includegraphics[width=\linewidth]{pictures/timothy/inbending_lessThan27.png}
%        \caption{}\label{fig:fig_a}
\end{subfigure}
\begin{subfigure}[t]{0.32\textwidth}
\centering
\includegraphics[width=\linewidth]{pictures/timothy/inbending_greaterThan27.png}
%        \caption{}\label{fig:fig_a}
\end{subfigure}
\begin{subfigure}[t]{0.32\textwidth}
\centering
\includegraphics[width=\linewidth]{pictures/timothy/outbending.png}
%        \caption{}\label{fig:fig_a}
\end{subfigure}
\begin{subfigure}[t]{0.32\textwidth}
\centering
\includegraphics[width=\linewidth]{pictures/timothy/outbending_lessThan27.png}
%        \caption{}\label{fig:fig_a}
\end{subfigure}
\begin{subfigure}[t]{0.32\textwidth}
\centering
\includegraphics[width=\linewidth]{pictures/timothy/outbending_greaterThan27.png}
%        \caption{}\label{fig:fig_a}
\end{subfigure}

%
%\begin{minipage}[t]{.4\textwidth}
\caption{The fit missing mass squared peak positions before (black) and after (red) proton energy loss corrections for the reaction $ep \rightarrow e' p \pi^+ X$. The top row corresponds to inbending data and the bottom row to outbending data. The first column is integrated over all angles, the second column are protons with polar angle below 27 degrees and the final column are protons with a polar angle above 27 degrees.}
\label{fig:hayward_results}
%\end{minipage}
\end{figure}

\FloatBarrier

